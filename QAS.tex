\documentclass{beamer}
\usepackage[UTF8,noindent]{ctexcap}
\usepackage{graphicx}
\title{Query Answer Selection}
\author{yixun}
\date{\today}
\subject{Query Answer Selection}
\keywords{Machine Learning, Deap Learning, CNN, RNN, LSTM}
\begin{document}
\begin{frame}
\titlepage
\end{frame}
\begin{frame}{任务描述}{Query Answer Selection是一个分类任务}
\begin{itemize}
\item 对于Query $q_i \in Q$,候选答案集合为$\{a_{i1}, a_{i2}, \dots, a_{im}\}$,对于每一个答案有一个标记值$y_{ij}$,$y_{ij} = 1$表示是问题$q_i$的答案,$y_{ij} = 0$表示不是$q_i$的答案。
\item 对于标注的所有三元组$(q_i, a_{ij}, y_(ij))$,任务就是要学习一个分类器,来预测正确的答案。
\end{itemize}
\end{frame}
\begin{frame}{任务描述}{一些例子}

\end{frame}
\begin{frame}{MengQiu Wang, 2007(CMU)}{What is the Jeopardy Model? A Quasi-Synchronous Grammar for QA}
\begin{itemize}
\item 基于语法结构相似度的模型
\item 使用WordNet的语义特征,同义词,反义词,上下谓词,形式曲折变化
\item 提供了TREC8-12/TREC8-13数据集用于训练测试
\item 评测:MAP/MRR
  \begin{itemize}
  \item 100人工判断:dev set: 0.6812/0.7636, test set: 0.6029/0.6852
  \item 2293 noisy: dev set: 0.6683/0.7443, test set: 0.5655/0.6687
  \end{itemize}
\end{itemize}
\end{frame}
\begin{frame}{Lei Yu, 2014(Google)}{Deep Learning for Answer Sentence Selection}
\begin{itemize}
\item 使用Bordes, 2014的公式衡量$a$是$q$正确回答的概率:
  $p(y=1|q,a)=\sigma(q^TMa+b)$
\item 学习目标:最小化交叉熵
\item 表示句子vector:
  \begin{itemize}
  \item Bag-of-words: word embedding加和的均值
  \item Bigram CNN
  \end{itemize}
\item 使用word count共现,idf * word count共现,上面训练的QA匹配概率训练LR模型
\item test set: 0.7113/0.7846
\end{itemize}
\end{frame}
\begin{frame}{Aliaksei Severyn, 2015(Google)}{Learning to Rank Short Text Pairs with Convolutional Deep Neural Networks}
\begin{itemize}
\item 模型:
\includegraphics[scale=0.3]{CNN.jpg}
\item test set:0.7459/0.8078
\item github: https://github.com/aseveryn/deep-qa,paper作者自己的实现
\end{itemize}
\end{frame}
\begin{frame}{Di Wang 2015(CMU)}{A Long Short-Term Memory Model for Answer Sentence Selection in Question Answering}
\end{frame}
\begin{frame}{Ming Tan 2016(IBM)}{LSTM-BASED DEEP LEARNING MODELS FOR NONFACTOID ANSWER SELECTION}
\end{frame}
\begin{frame}{总结}
\end{frame}
\end{document}